\documentclass[10pt,a4paper]{article}
\usepackage[paper=a4paper, hmargin=1.5cm, bottom=1.5cm, top=2cm]{geometry}

\usepackage[utf8x]{inputenc}
\usepackage[spanish]{babel}

\usepackage{mathtools}
\usepackage{amsmath}
\usepackage{amsfonts}
\usepackage{amssymb}

\usepackage{color}

\usepackage{xcolor}
\usepackage{listingsutf8}
\usepackage{booktabs}
\usepackage{hyperref}
\usepackage{multirow}

\usepackage{caption}
\usepackage{subcaption}

\usepackage{algorithm}
\usepackage{algpseudocode}

\usepackage{graphicx}
\usepackage{tikz}
\usepackage{relsize}
\usepackage{epstopdf}

\DeclarePairedDelimiter{\ceil}{\lceil}{\rceil}

%\let\NombreFuncion=\textsc
%\let\TipoVariable=\texttt

%\newcommand{\TipoFuncion}[3]{%
  %\NombreFuncion{#1}(#2) \ifx#3\empty\else $\to$ \res\,: \TipoVariable{#3}\fi%
%}

% set the default code style
\lstset{
    frame=tb, % draw a frame at the top and bottom of the code block
    tabsize=4, % tab space width
    showstringspaces=false, % don't mark spaces in strings
    numbers=left, % display line numbers on the left
    commentstyle=\color{green}, % comment color
    keywordstyle=\color{blue}, % keyword color
    stringstyle=\color{red} % string color
}

% mathy stuff
\newtheorem{theorem}{Theorem}[section]
\newtheorem{lemma}[theorem]{Lemma}
\newtheorem{proposition}[theorem]{Proposición}
\newtheorem{corollary}[theorem]{Corollary}

\newenvironment{proof}[1][Demostración]{\begin{trivlist}
\item[\hskip \labelsep {\bfseries #1}]}{\end{trivlist}}
\newenvironment{definition}[1][Definición]{\begin{trivlist}
\item[\hskip \labelsep {\bfseries #1}]}{\end{trivlist}}
\newenvironment{example}[1][Example]{\begin{trivlist}
\item[\hskip \labelsep {\bfseries #1}]}{\end{trivlist}}
\newenvironment{remark}[1][Remark]{\begin{trivlist}
\item[\hskip \labelsep {\bfseries #1}]}{\end{trivlist}}

\newcommand{\qed}{\nobreak \ifvmode \relax \else
      \ifdim\lastskip<1.5em \hskip-\lastskip
      \hskip1.5em plus0em minus0.5em \fi \nobreak
      \vrule height0.75em width0.5em depth0.25em\fi}

\title{Investigación Operativa \\ Coloreo Particionado de Grafos}

\newcommand{\order}[1]{$\mathcal{O}(#1)$}

\begin{document}

%% cover page

\maketitle

\bigskip

\begin{table}[h]
\centering
\begin{tabular}{|l l l|}
\hline
Integrantes       & \multicolumn{1}{c}{LU}     & Correo electrónico        \\ \hline
Martin Baigorria & \multicolumn{1}{c}{575/14} & martinbaigorria@gmail.com \\ 
Andrés Armesto Brosio & 512/14 & andresarmesto@gmail.com \\  \hline
\end{tabular}
\end{table}

\vfill
\textbf{Resumen:} El presente trabajo práctico tiene como objetivo resolver el problema del coloreo particionado de grafos utilizando programación lineal. Para ello, en un primer momento modelamos el problema y lo implementamos utilizando CPLEX. Se experimenta con diferentes configuraciones de los métodos Branch \& Bound y Branch \& Cut, evaluando diferentes heurísticas, configuraciones y variando los tipos de grafos a resolver.

\textbf{Keywords:} Linear Programming, Partitioned Graph Coloring, Branch \& Bound, Cut \& Branch, CPLEX.

\newpage
\tableofcontents
\newpage

% end cover page

\section{Modelo}

Dado un grafo $G(V,E)$ con $n = |V|$ vertices y $m = |E|$ aristas,  un coloreo de G se define como una asignacion de un color o etiqueta a cada $v \in V$ de forma tal que para todo  par de vertices adyacentes $(p,q) \in E$ poseen colores distintos. El clasico problema de \textit{coloreo de grafos} consiste en encontrar un coloreo del grafo que utilize la menor cantidad de colores posibles.

En este trabajo resolveremos una variante de este problema, el \textit{coloreo particionado de grafos}. A partir de un conjunto de vertices $V$ que se encuentra particionado en $V_1,...,V_k$, el problema consiste en asignar un color $c \in C$ a solo un vertice de cada particion de forma tal que dos vertices adyacentes no reciban el mismo color y minimizando la cantidad de colores utilizados.

Este problema se puede modelar con Programacion Lineal Entera. Para ello, definamos las siguientes variables:

\hspace{1px}

\begin{center}
$x_{pj} = \begin{cases}
  1 & \text{si el color $j$ es asignado al vertice $p$} \\
  0 & \text{en caso contrario}
\end{cases}$

\hspace{1px}

$w_j = \begin{cases}
  1 & \text{si $x_{pj} = 1$ para algun vertice $p$} \\
  0 & \text{en caso contrario}
\end{cases}$
\end{center}

\subsection{Funcion objetivo}

De esta forma la funcion objetivo del LP consiste en minimizar la cantidad de colores utilizados:
\begin{equation}
min \sum_{j \in C} w_j
\end{equation}

Notar que $|C|$ esta acotado superiormente por la cantidad de particiones $k$.

\vspace{10px}

\subsection{Restricciones}

Los vertices adyacentes no comparten color. Recordar que no necesariamente se le asigna un color a todo vertice.
\begin{equation}
x_{ij} + x_{kj} \leq 1 \;\;\;\;\; \forall (i,k) \in E,\;\; \forall j \in C
\end{equation}

Solo se le asigna un color a un unico vertice de cada particion $p \in P$. Esto implica que cada vertice tiene a lo sumo solo un color.
\begin{equation}
\sum_{i \in V_p} \sum_{j \in C} x_{ij} = 1 \;\;\;\;\; \forall p \in P
\end{equation}

Si un nodo usa color $j$, $w_j = 1$:
\begin{equation}
x_{ij} \leq w_j \;\;\;\;\; \forall i \in V, \forall j \in C
\end{equation}

Integralidad y positividad de las variables:
\begin{equation}
x_{ij} \in \{0,1\} \;\;\;\;\; \forall i \in V, \forall j \in C
\end{equation}

\begin{equation}
w_j \in \{0,1\} \;\;\;\;\; \forall j \in C
\end{equation}
\newpage
\section{Branch \& Bound}

Los algoritmos de tipo Branch \& Bound se utilizan para resolver problemas de programación lineal entera (PLE) o programación lineal entera mixta (PLEM, PEM, o MIP en inglés). El algoritmo consiste en tomar el problema de PEM y resolver en primera instancia la relajación lineal, es decir aquella que relaja las condiciones de integralidad sobre las variables. Con esto se obtiene un $x^*$, es decir una solución óptima de la relajación lineal. Si la solución obtenida cumple con las condiciones de integralidad, se ha encontrado un óptimo y el algoritmo termina. Si existe al menos una variable que no la cumple, se parte el problema original en 2 o más subproblemas. El proceso de partición se llama \textit{branching}, y existe diversidad de criterios para realizarlo. Sin embargo, todas las formas de branching cumplen que todos los puntos factibles del problema original deben estar en alguna partición, y que el $x^*$ hallado no pertenece a ninguna de ellas, de forma de no caer nuevamente en él. El proceso se repite en cada subproblema que nace, y termina cuando no quedan nodos por explorar. A su vez, otra parte importante del algoritmo consiste en cortar aquellas ramas cuyo valor óptimo de la relajación lineal es peor que el valor óptimo obtenido hasta ese momento. A este fenómeno se lo llama poda, o \textit{bounding} en inglés.

Los criterios más importantes a determinar en un algoritmo de Branch \& Bound son cómo realizar el branching (qué variables) y qué nodos a explorar, o cómo recorrer el árbol de enumeración (verticalmente, horizontalmente, etc.).

La implementación del modelo y del Branch \& Bound se encuentran en el apéndice.
\section{Desigualdades}

\subsection{Desigualdad de Clique}

Sea $j_0 \in \{1,...,n\}$ y sea $K$ una clique maximal de $G$. La desigualdad clique estan definida por:

\begin{equation}
\sum_{p \in K} x_{pj_0} \leq w_{j_0}
\end{equation}

\begin{proof}
Para esta demostracion utilizaremos las desigualdades Chvátal-Gomory sobre las restricciones del LP planteado en la seccion \ref{restricciones} e induccion. A priori el teorema es bastante intuitivo. Si pinto algun vertice de una clique, no puedo pintar ninguno adyacente del mismo color sin importar la forma en la que particione los vertices del grafo. Sea $n$ el tamanio de la clique maximal.

\hfill

\textbf{Casos Base}
\begin{enumerate}
\item $n=1$: Si en la clique maximal tengo solo un vertice, no existe arista que contenga este vertice, caso contrario la clique tendria dos elementos. Por lo tanto, este vertice puede estar pintado o no dentro de la particion. Es decir, se cumple la ecuacion que queremos probar.
\item $n=2$: Si la clique maximal tiene dos elementos, por definicion son conexos. Por la restriccion que indica que los vertices adyacentes no comparten color, aqui hay 2 opciones. La primera opcion es que a ningun vertice se le asigna un color $j_0$. La otra opcion es que dada la estructura de particiones, se le asigne solo a uno de ellos el color $j_0$. Por lo tanto la desigualdad para $n=2$ vale.
\item $n=3$: Este es el caso mas interesante en el que utilizamos la desigualdad de Chvátal-Gomory. Si la clique tiene 3 vertices, hay tres desigualdades que se deben cumplir:

\begin{itemize}
\item $x_{1j_0} + x_{2j_0} \leq 1$
\item $x_{2j_0} + x_{3j_0} \leq 1$
\item $x_{1j_0} + x_{3j_0} \leq 1$
\end{itemize}

Multiplicando todas estas desigualdades por $1/3$ y sumando entonces:

$1/3 (x_{1j_0} + x_{2j_0})  + 1/3 (x_{2j_0} + x_{3j_0}) + 1/3 (x_{2j_0} + x_{3j_0}) \leq 3/2$

Como $x_{ij}$ toma valores enteros, entonces:
$1/3 (x_{1j_0} + x_{2j_0})  + 1/3 (x_{2j_0} + x_{3j_0}) + 1/3 (x_{2j_0} + x_{3j_0}) \leq 1$

Simplificando: $x_{1j_0} + x_{2j_0} +  x_{3j_0} \leq 1$.

Utilizando la definicion de $w_j$ entonces: $x_{1j_0} + x_{2j_0} +  x_{3j_0} \leq w_{j_0}$

Por lo tanto la desigualdad vale para $n=3$.

\end{enumerate}

\hfill

\textbf{Paso Inductivo:} $P(n-1) \implies P(n)$

Como vale la hipotesis inductiva, sabemos que:

\begin{equation*}
\sum_{p \in K-n} x_{pj_0} \leq w_{j_0}
\end{equation*}

Al agregar un vertice a la clique, agregamos $n-1$ aristas:

$x_{1j_0} + x_{nj_0} \leq 1$, $x_{2j_0} + x_{nj_0} \leq 1$,...,
$x_{(n-1)j_0} + x_{nj_0} \leq 1$

Utilizando esto, podemos ver que:

\begin{equation*}
x_{nj_0} + \sum_{p \in K-n} x_{pj_0} \leq w_{j_0}
\end{equation*}

Esto es claramente equivalente a lo que queremos demostrar y se puede justificar a partir de dos casos:

\begin{itemize}
\item Si al vertice $x_{nj_0}$ se le asigna un color, por las restricciones de las aristas que agregamos al resto de los vertices de la clique no se le puede asignar el color $j_0$.
\item Si al vertice $x_{nj_0}$ no se le asigna un color o se le asigna un color diferente a $j_0$, por hipotesis inductiva sabemos que lo que queremos probar vale. \hfill $\square$
\end{itemize}
\end{proof}

\subsection{Desigualdad de Aujero Impar}

Sea $j_0 \in \{1,...,n\}$ y sea $C_{2k+1} = v_1,...,v_{2k+1}$, $k \geq 2$, un aujero de longitud impar. La desigualdad esta definida por:

\begin{equation}
\sum_{p \in C_{2k+1}} x_{pj_0} \leq k w_{j0}
\end{equation}

\begin{proof}
Por teoremas de coloreo (que se prueban en general por induccion), sabemos que el numero cromatico $\chi(C) = 3$. En el peor de los casos, cada vertice del aujero estara en una particion diferente. Aqui nuevamente tenemos dos casos:

\begin{itemize}
\item Si no se asigna el color $j_0$ a algun vertice del aujero, la desigualdad vale.
\item Si se asigna el color $j_0$, en el peor de los casos el mismo sera utilizado por a lo sumo $(|C|-1)/2$ vertices. Como $|C| = 2k+1$,  $(2k+1-1)/2 = k$. Por lo tanto vale la desigualdad.  \hfill $\square$
\end{itemize}

\end{proof}

\subsection{Planos de Corte}

Luego de relajar el PLEM, los algoritmos de separacion buscan acotar el espacio de busqueda para que se parezca mas a la capsula convexa. Existen algoritmos de separacion exactos y heuristicos. Los algoritmos heuristicos, luego de resolver la relajacion del problema entero y encontrar una solucion optima $x^*$, retornan una o mas desigualdades de la clase violadas por alguna familia de desigualdades. Por ser un algoritmo heuristico, es posible que exista una desigualdad de la clase violada aunque el procedimiento no sea capaz de encontrarla. Si se encuentra una desigualdad que es violada por la solucion optima de la relajacion, se agrega esta nueva restriccion y se vuelve a resolver el programa lineal. Este procedimiento se conoce como algoritmo de plano de corte. Si una solucion optima al problema existe, este tipo de algoritmo no necesariamente la encuentra. Por ejemplo, las heuristicas que encuentran desigualdades validas pueden fallar y el algoritmo no puede continuar.

\subsubsection{Heuristica de Separacion para Clique}

\subsubsection{Heuristica de Separacion para Aujero Impar}
\newpage
\section{Experimentación}

Dada la cantidad de vértices, los grafos se generan en el formato estándar DIMACS \footnote{Para ver algunos ejemplos del formato: http://mat.gsia.cmu.edu/COLOR/instances.html}. El generador toma como parámetro la densidad del grafo. Dada una clique con esa cantidad de vértices, se elijen vértices al azar hasta que se llega a la densidad deseada. Debido a que estas instancias están diseñadas para coloreo de grafos, asignamos los vértices de forma uniforme en el total de particiones pasado por parámetro a nuestro programa de coloreo particionado.

Por cuestiones de tiempo, cada uno de los experimientos CPLEX fue ejecutado sin límite de cantidad de threads, con un procesador Intel(R) Core(TM) i7-3610QM CPU @ 2.30GHz y 16GB de memoria RAM.

\subsection{Eliminación de simetría}

Al igual que el problema de coloreo de grafos, el problema del coloreo particionado de grafos presenta una gran cantidad de soluciones simétricas. De no romper la simetría del problema, los algoritmos tendrían un espacio de búsqueda mucho mayor, moviéndose por soluciones que, siendo computacionalmente distintas, en la práctica se trata de la misma. Esto afecta el tiempo de ejecución de forma considerable a medida que crece el tamaño del problema. Para romper la simetría en nuestro problema, en la sección \ref{simetria} mostramos cómo utilizamos la clásica condicion de coloreo de que los colores se deben utilizar en orden. Este fenómeno se puede ver en el siguiente gráfico:

\begin{figure}[h]
\centering
\includegraphics[scale=0.7]{img/2-symmetry_v25_p5_l40_t1_b0.png}
\caption{Tiempo de resolución del modelo incluyendo o no eliminación de simetría.}
\end{figure}

Esto nos brinda la noción sumamente relevante de la importancia y efectividad de romper simetría al realizar la formulación de un LP. Cabe mencionar que existen muchas otras estrategias o expresiones para disminuir aun más el grado de simetría de la formulación. La escogida bajo ninguna circunstancia debe ser considerada la mejor posible.

\pagebreak

\subsection{Efectividad de las familias de desigualdades}

La idea de este experimento es comparar las diferentes estrategias de planos de corte. Para ello, se eligió a 40 como la cantidad de cortes de cada tipo que se podían agregar, con una sola iteración:
% en cada iteración se podían agregar hasta 40 cortes?

\begin{figure}[h]
  \centering
  \begin{minipage}[b]{0.49\textwidth}
    \includegraphics[width=\textwidth]{img/5-cuts_v40_p20_i1_l40_t1_b0.png}
    \caption{Estrategias de planos de corte (tiempo)}
  \end{minipage}
  \hfill
  \begin{minipage}[b]{0.49\textwidth}
    \includegraphics[width=\textwidth]{img/5-cuts_v40_p20_i1_l40_t1_b0_nodes.png}
    \caption{Estrategias de planos de corte (nodos recorridos)}
  \end{minipage}
\end{figure}

Lo primero que podemos observar es que no siempre hay una estrategia ganadora por sobre las otras. Se observa con claridad una dependencia entre la densidad del grafo y la estrategia que tuvo mejores resultados. Cuanto más denso, más cliques nuestra heurística debería encontrar, y a priori uno esperaría que los tiempos mejoren. Esto no sucede, de hecho agregar las restriciones de clique empeora el tiempo de ejecución con respecto al resultado de utilizar B\&B. También podemos observar que un mejor tiempo de ejecución no necesariamente implica que se recorren menos nodos en árbol de enumeración. En contra de lo que esperábamos inicialmente, las desigualdades de agujero impar parecen funcionar bien, aunque por supuesto esto se podría constatar con mayor peso de llevar a cabo una experimentación mas exhaustiva.

\subsection{Efecto de aumentar el número de particiones}

A medida que aumentamos el número de particiones, el problema comienza a parecerse más a uno de coloreo. Dado que las desigualdades que implementamos son clásicas de coloreo, es de esperar que la performance mejore a medida que aumenta el número de particiones \cite{coloring}. Para Cut \& Branch, sólo utilizamos los mejores 40 cortes de clique con una iteración. A medida que aumenta el número de particiones, podemos observar cómo la ganancia del corte es mayor.

\begin{figure}[h]
  \centering
  \begin{minipage}[b]{0.49\textwidth}
    \includegraphics[width=\textwidth]{img/3-partitions_v30_d50_i1_co0_l40_t1_b0.png}
    \caption{Tiempo de ejecucion a medida que aumenta el numero de particiones.}
  \end{minipage}
  \hfill
  \begin{minipage}[b]{0.49\textwidth}
    \includegraphics[width=\textwidth]{img/3-partitions_v30_d50_i1_co0_l40_t1_b0_nodes.png}
    \caption{Nodos recorridos a medida que aumenta el numero de particiones.}
  \end{minipage}
\end{figure}

\pagebreak

\subsection{Efecto de aumentar la densidad del grafo}

\begin{figure}[h]
  \centering
  \begin{minipage}[b]{0.49\textwidth}
    \includegraphics[width=\textwidth]{img/1-bb_vs_bc_v20_p10_i1_co0_l40_t1_b0.png}
  \end{minipage}
  \hfill
  \begin{minipage}[b]{0.49\textwidth}
    \includegraphics[width=\textwidth]{img/1-bb_vs_bc_v40_p10_i1_co0_l40_t1_b0.png}
  \end{minipage}
  \begin{minipage}[b]{0.49\textwidth}
    \includegraphics[width=\textwidth]{img/1-bb_vs_bc_v60_p10_i1_co0_l40_t1_b0.png}
  \end{minipage}
  \hfill
  \begin{minipage}[b]{0.49\textwidth}
    \includegraphics[width=\textwidth]{img/1-bb_vs_bc_v100_p10_i1_co0_l40_t1_b0.png}
  \end{minipage}
	\caption{Efecto de aumentar la densidad del grafo.}
\end{figure}

A medida que aumenta la densidad del grafo, el problema de coloreo se vuelve sin duda más difícil. En los casos donde el número de particiones es mayor en relación al numero de vértices, Branch \& Cut con 1 iteración y 40 desigualdades violadas parece funcionar mejor. Esto no sucede en grafos esparsos, donde Branch \& Bound puro tiene un menor tiempo de ejecución.

\pagebreak

\subsection{Efecto de aumentar la cantidad de restricciones incorporadas por iteración}

Para todos nuestros experimentos en general utilizamos sólo 1 iteración con un límite de 40 desigualdades por familia. La idea de este experimento es evaluar esta configuración. Para ello, utilizamos un grafo con 40 vértices y 20 particiones.

\begin{figure}[h]
  \centering
  \begin{minipage}[b]{0.49\textwidth}
    \includegraphics[width=\textwidth]{img/6-thresholds_v40_p20_i1_t1_b0.png}
    \caption{Tiempo de ejecución al incrementar el número de restricciones incorporadas.}
  \end{minipage}
  \hfill
  \begin{minipage}[b]{0.49\textwidth}
    \includegraphics[width=\textwidth]{img/6-thresholds_v40_p20_i1_co2_t1_b0_nodes.png}
    \caption{Nodos recorridos al incrementar el número de restricciones incorporadas.}
  \end{minipage}
\end{figure}

Como podemos observar, agregar más restricciones no es siempre ventajoso. En un principio, agregar restricciones parece mejorar la ejecución del C\&B, pero ya a partir de 40 el tiempo de ejecución empeora de forma abrupta para las cliques. Esto no sucede para las restricciones de agujero impar. Nuevamente, esto se puede deber a que nuestra heurística de clique no es lo suficientemente buena.

\subsection{Efecto de aumentar la cantidad de iteraciones de planos de corte}

\begin{figure}[h]
\centering
\includegraphics[scale=0.5]{img/7-iterations_v40_p10_l40_t1_b0.png}
\caption{Tiempo de ejecución al aumentar la cantidad de iteraciones de planos de corte.}
\end{figure}

Como podemos ver, aumentar el numero de iteraciones de planos de corte no necesariamente mejora el tiempo de ejecución. En cada iteración lo que hacíamos era generar una familia en función de la solución de la relajación del problema, y luego agregar las \textit{mejores} restricciones. En relación a la sección anterior, esto también esta relacionado con el $threshold$ que elegimos para hacer la experimentación.

\pagebreak

\subsection{Comparación B\&B, C\&B, CPLEX default}

\begin{figure}[h]
  \centering
  \begin{minipage}[b]{0.49\textwidth}
    \includegraphics[width=\textwidth]{img/8-compare_v20_p10_i1_l40_t1_b0.png}
  \end{minipage}
  \hfill
  \begin{minipage}[b]{0.49\textwidth}
    \includegraphics[width=\textwidth]{img/8-compare_v20_p20_i1_l40_t1_b0.png}
  \end{minipage}
  \begin{minipage}[b]{0.49\textwidth}
    \includegraphics[width=\textwidth]{img/8-compare_v40_p10_i1_l40_t1_b0.png}
  \end{minipage}
  \hfill
  \begin{minipage}[b]{0.49\textwidth}
    \includegraphics[width=\textwidth]{img/8-compare_v40_p20_i1_l40_t1_b0.png}
  \end{minipage}
	\caption{Comparacion B\&B, C\&B, CPLEX default para diferentes grafos.}
\end{figure}

Dado que el CPLEX por default utiliza cortes de Gomory y preprocesamiento de variables, no nos sorprende que en general sea superior a nuestras otras estrategias para grafos densos. Una propuesta interesante podría ser repetir esta experimentación permitiendo los cortes y el preprocesamiento para todas nuestras estrategias. Otra observación, el gráfico superior derecho es el caso de coloreo de grafos, dado que cada vértice pertenece a una partición diferente. Aquí podemos ver que las desigualdades de clique son sumamente útiles. 

\pagebreak

\subsection{Estrategias de recorrido del árbol de enumeración
y selección de variable de branching}

Existen muchas estrategias de recorrido del árbol de enumeración. En este trabajo solo analizaremos DFS y BBS. DFS (Depth First Search) recorre el árbol de enumeración de B\&B primero en profundidad. Por otro lado, BBS (Best Bound Search) recorre el árbol de enumeración utilizando alguna estrategia para intentar buscar una buena cota lo mas rápido posible. En general se utilizan estrategias heurísticas. En el caso de CPLEX, dado un nodo padre se calcula la solución a la relajación de todos sus hijos y luego se continua recorriendo el nodo con el mayor resultado de la función objetivo. \footnote{http://www-01.ibm.com/support/knowledgecenter/SSSA5P\_12.6.1/ilog.odms.cplex.help/CPLEX/Parameters/topics/NodeSel.html}.

Ambas estrategias son sumamente ventajosas ya que permiten obtener una cota superior a la solución final para utilizar de poda al hacer backtracking sobre el árbol de enumeración. Dado que no utilizamos heurísticas iniciales, esta estrategia parece razonable. 

Por otro lado, las estrategias de selección de variable buscan encontrar cual es la mejor variable sobre la cual hacer branching. Hay muchas reglas, como por ejemplo \textit{max/min infeasibility}. Mientras que la regla de \textit{minimum infeasibility} busca hacer branching sobre mas cercana al entero, la regla de \textit{maximum infeasibility} busca hacer exactamente lo contrario \footnote{http://www-01.ibm.com/support/knowledgecenter/SS9UKU\_12.4.0/com.ibm.cplex.zos.help/Parameters/topics/VarSel.html}.

En esta sección analizaremos 4 combinaciones de estrategias de recorrido del árbol de enumeración y selección de variable de branching para B\&B puro y C\&B con cortes de clique, 1 iteración y $threshold = 30$. Las combinaciones que analizaremos son: DFS + MAXINFEAS, DFS + MININFEAS, BESTBOUND + MAXINFEAS, BESTBOUND + MININFEAS.

\begin{figure}[h]
  \centering
  \begin{minipage}[b]{0.49\textwidth}
    \includegraphics[width=\textwidth]{img/9-tree_v60_p10_i1_l30_s1.png}
  \end{minipage}
  \hfill
  \begin{minipage}[b]{0.49\textwidth}
    \includegraphics[width=\textwidth]{img/9-tree_v60_p10_i1_l30_s1_nodes.png}
  \end{minipage}
  \begin{minipage}[b]{0.49\textwidth}
    \includegraphics[width=\textwidth]{img/9-tree_v60_p10_i1_l30_s2.png}
  \caption{Tiempo de ejecución dependiendo de la estrategias de recorrido y selección de variable.}
  \end{minipage}
  \hfill
  \begin{minipage}[b]{0.49\textwidth}
    \includegraphics[width=\textwidth]{img/9-tree_v60_p10_i1_l30_s2_nodes.png}
  \caption{Nodos recorridos dependiendo de la estrategias de recorrido y selección de variable.}
  \end{minipage}
\end{figure}

Como podemos observar, en general C\&B tiene tiempos de ejecución menores y a su vez recorre menos nodos. La mejor estrategia para este problema parece ser BESTBOUND + MAXINFEAS. CPLEX utiliza por default BESTBOUND, aunque utiliza una heurística para elegir la variable de branching. 

\subsection{Instancias DIMACS}

Las instancias DIMACS son comúnmente utilizadas en la literatura como instancias de benchmarking. A continuación mostramos nuestros tiempos de ejecución con B\&B y B\&C utilizando 1 iteración y $threshold = 30$ con solo desigualdades de clique. A su vez, ambas utilizan el recorrido del árbol de enumeración dado por default en CPLEX.

Cada tabla utiliza un numero de particiones diferente. Dado que las instancias DIMACS no tienen un numero de partición debido a que se utilizan normalmente para coloreo, asignamos uno nosotros y luego dividimos los vértices en orden en las diferentes particiones de forma uniforme.

Tomamos como tiempo de ejecucion limite 10 minutos, y reportamos el numero de colores encontrado por B\&B hasta ese momento. En todos los casos, B\&B y B\&C coincidieron con el numero de colores utilizados, por lo que los reportamos solo en una columna.

\begin{table}[H]
\centering
\caption{Benchmark con 10 particiones}
\begin{tabular}{|l|r|r|r|r|r|}
\hline
Problem & \multicolumn{1}{l|}{n} & \multicolumn{1}{l|}{m} & \multicolumn{1}{l|}{Tiempo tomado por B\&B (secs)} & \multicolumn{1}{l|}{Tiempo tomado por B\&C (secs)} & \multicolumn{1}{l|}{Colores utilizados} \\ \hline
anna & 138 & 493 & 0.04 & 0.50 & 1 \\ \hline
david & 87 & 406 & 0.02 & 0.29 & 1 \\ \hline
fpsol2.i.1 & 496 & 11654 & 0.51 & 8.71 & 1 \\ \hline
fpsol2.i.2 & 451 & 8691 & 0.44 & 7.66 & 1 \\ \hline
fpsol2.i.3 & 425 & 8688 & 0.45 & 8.15 & 1 \\ \hline
games120 & 120 & 638 & 0.04 & 0.32 & 1 \\ \hline
homer & 561 & 1629 & 0.11 & 0.72 & 1 \\ \hline
huck & 74 & 301 & 0.02 & 0.10 & 1 \\ \hline
inithx.i.1 & 864 & 18707 & 0.79 & 9.66 & 1 \\ \hline
inithx.i.2 & 645 & 13979 & 0.59 & 6.53 & 1 \\ \hline
inithx.i.3 & 621 & 13969 & 0.59 & 6.16 & 1 \\ \hline
jean & 80 & 254 & 0.02 & 0.10 & 1 \\ \hline
le450\_15a & 450 & 8168 & 0.31 & 14.68 & 1 \\ \hline
le450\_15b & 450 & 8169 & 0.35 & 15.68 & 1 \\ \hline
le450\_15c & 450 & 16680 & 0.64 & 36.05 & 1 \\ \hline
le450\_15d & 450 & 16750 & 0.83 & 38.00 & 1 \\ \hline
le450\_25a & 450 & 8260 & 0.34 & 24.95 & 1 \\ \hline
le450\_25b & 450 & 8263 & 0.33 & 15.41 & 1 \\ \hline
le450\_25c & 450 & 17343 & 0.70 & 42.17 & 1 \\ \hline
le450\_25d & 450 & 17425 & 0.70 & 39.60 & 1 \\ \hline
le450\_5a & 450 & 5714 & 0.27 & 8.19 & 1 \\ \hline
le450\_5b & 450 & 5734 & 0.30 & 13.78 & 1 \\ \hline
le450\_5c & 450 & 9803 & 0.46 & 29.12 & 1 \\ \hline
le450\_5d & 450 & 9757 & 0.47 & 33.13 & 1 \\ \hline
miles1000 & 128 & 3216 & 8.81 & 7.77 & 2 \\ \hline
miles1500 & 128 & 5198 & 10 min & 10 min & 3 \\ \hline
miles250 & 128 & 387 & 0.03 & 0.24 & 1 \\ \hline
miles500 & 128 & 1170 & 0.06 & 0.54 & 1 \\ \hline
miles750 & 128 & 2113 & 0.32 & 2.73 & 1 \\ \hline
mulsol.i.1 & 197 & 3925 & 0.15 & 2.20 & 1 \\ \hline
mulsol.i.2 & 188 & 3885 & 0.14 & 2.01 & 1 \\ \hline
mulsol.i.3 & 184 & 3916 & 0.16 & 3.02 & 1 \\ \hline
mulsol.i.4 & 185 & 3946 & 0.15 & 4.37 & 1 \\ \hline
mulsol.i.5 & 186 & 3973 & 0.15 & 3.42 & 1 \\ \hline
myciel2 & 32766 & 0 & 4.37 & 110.76 & 1 \\ \hline
myciel3 & 11 & 20 & 0.01 & 0.01 & 3 \\ \hline
myciel4 & 23 & 71 & 0.01 & 0.02 & 1 \\ \hline
myciel5 & 47 & 236 & 0.01 & 0.04 & 1 \\ \hline
myciel6 & 95 & 755 & 0.04 & 0.29 & 1 \\ \hline
myciel7 & 191 & 2360 & 0.09 & 1.75 & 1 \\ \hline
queen10\_10 & 100 & 1470 & 0.12 & 0.61 & 1 \\ \hline
queen11\_11 & 121 & 1980 & 0.18 & 1.03 & 1 \\ \hline
queen12\_12 & 144 & 2596 & 0.36 & 2.29 & 1 \\ \hline
queen13\_13 & 169 & 3328 & 0.44 & 2.73 & 1 \\ \hline
queen14\_14 & 196 & 4186 & 0.18 & 5.05 & 1 \\ \hline
queen15\_15 & 225 & 5180 & 0.23 & 7.02 & 1 \\ \hline
queen16\_16 & 256 & 6320 & 0.23 & 8.62 & 1 \\ \hline
queen5\_5 & 25 & 160 & 0.06 & 0.14 & 3 \\ \hline
queen6\_6 & 36 & 290 & 0.12 & 0.54 & 2 \\ \hline
queen7\_7 & 49 & 476 & 0.12 & 0.41 & 2 \\ \hline
queen8\_12 & 96 & 1368 & 3.54 & 3.96 & 2 \\ \hline
queen8\_8 & 64 & 728 & 0.26 & 0.64 & 2 \\ \hline
queen9\_9 & 81 & 1056 & 1.76 & 2.54 & 2 \\ \hline
school1 & 385 & 19095 & 1.17 & 90.04 & 1 \\ \hline
school1\_nsh & 352 & 14612 & 0.83 & 46.13 & 1 \\ \hline
zeroin.i.1 & 211 & 4100 & 0.16 & 4.87 & 1 \\ \hline
zeroin.i.2 & 211 & 3541 & 0.17 & 3.82 & 1 \\ \hline
zeroin.i.3 & 206 & 3540 & 0.25 & 1.91 & 1 \\ \hline
\end{tabular}
\end{table}



\newpage
\section{Conclusión}

El famoso problema de coloreo de grafos, que ha sido estudiado ampliamente en la literatura, es un caso particular del coloreo particionado de grafos, donde cada vértice pertenece a una partición diferente. Por esta razón, en primer lugar notamos que el problema del coloreo particionado sería al menos tan difícil como el problema de coloreo estándar, que de por sí representa un problema complicado.

Las desigualdades de planos de cortes que se han implementado en este trabajo son desigualdades utilizadas normalmente para coloreo. Por esta razón, a lo largo de todo el trabajo hemos notado que los algoritmos de Cut \& Branch tienden a funcionar mejor en los casos en que la relación cantidad de particiones sobre cantidad de nodos es mayor. La intuición nos sugiere que deben existir mejores familias, que exploten el hecho de que el grafo esté dividido en particiones, si bien encontrarlas y desarrollarlas escape del alcance de este trabajo.

Uno de los primeros problemas que encontramos al programar el algoritmo de Cut \& Branch fue encontrar buenas heurísticas para las familias de desigualdades que probamos válidas. A priori sabíamos, por ejemplo, que generar todas las cliques de manera exhaustiva es un problema NP-Hard. Aquí es donde interviene sin duda la creatividad del investigador para el diseño de los algoritmos, y obtener una heurística que genere un buen resultado. A lo largo de este trabajo probamos varias estrategias, y finalmente nos quedamos con una que depende de la solución de la relajación en cada iteración. Sin embargo, no tenemos ninguna duda de que existen heurísticas mucho más efectivas. A su vez, una vez encontrado este conjunto de desigualdades violadas, es sumamente importante establecer un criterio para decidir cuáles deben ser agregadas al programa lineal. Se pudo comprobar, en términos generales, que agregarlas todas hace que la optimización sea más lenta.

Por otro lado, en nuestro caso hemos notado que el tiempo de cómputo no está dominado por la generación de estas familias, sino por la resolución del programa lineal. La explicación más probable de esto es que el algoritmo de Cut \& Branch implementado realiza cortes únicamente en el nodo raíz (una o más iteraciones), por lo que la cantidad de llamados a las heurísticas no crece al recorrer el árbol. En el caso más general, esto no se cumple, y es tarea del investigador procurar un balance entre el tiempo de ejecución de la heurística y el tiempo de ejecución necesario para resolver el programa lineal. Por supuesto, también es importante descubrir en qué tipos de problemas (tamaño, características particulares, etc) conviene utilizar determinadas estrategias como cortes específicos, o diferentes modos de recorrer el árbol de enumeración. En definitiva, el diseño del algoritmo requiere conocer las distintas variables intervinientes, y cómo éstas afectan el resultado final. Es por ello que las horas experimentación con diversas instancias son una etapa clave para conocer el proceso y poder refinarlo. Posteriormente, se podrá acomodarlo o refinarlo, para obtener buenos resultados ya sea en una instancia en particular, o en términos más generales de aplicación.

A lo largo de este trabajo, la performance de CPLEX nos sorprendió notablemente. Por default, CPLEX en sí funciona bastante bien. Conseguir una mejor descripción de la cápsula convexa muchas veces es difícil, y \textit{en términos prácticos} puede llegar a no valer la pena. Debe balancearse el esfuerzo y la dificultad con el tiempo de cómputo y la calidad de la solución obtenida. Por supuesto, también influye el tamaño de instancia que uno necesite resolver. Aquí es donde entra CPLEX, que con una formulación simple del PPL logra optimizar el problema relativamente bien para instancias razonablemente chicas. Como comentario, comprobamos la suma importancia de romper con la simetría de los problemas. Esto en general no representa gran dificultad adicional, y mejora los tiempos de ejecución de forma considerable.

En cuanto a oportunidades de mejora, se podría generalizar el algoritmo Branch \& Cut, agregando heurísticas iniciales y primales (qué mejoran el cálculo de cotas, la inicialización y la poda de ramas), o encontrando mejores estrategias de cortes en diferentes partes del árbol, utilizando los \textit{callbacks} de CPLEX. Asimismo, podrían incluirse estrategias de preprocesamiento.

Existe gran cantidad de parámetros a calibrar para lograr una buena performance, y su elección en general se basa en una experimentación que logre emular los casos más comunes en la práctica. Durante este trabajo, no se experimentó en profundidad con instancias y problemas sumamente difíciles debido al tiempo acotado para realizar el mismo. Sería interesante, sin embargo, ver hasta qué punto pueden llegar los algoritmos con una buena configuración.

\vfill

\bibliographystyle{plain}
\bibliography{bibliografia}
\pagebreak

\section{Apéndice A: Código}
\subsection{coloring.cpp}
\lstinputlisting[language=C++, breaklines=true]{../src/coloring.cpp}
\end{document}